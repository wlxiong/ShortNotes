%%%%%%%%%%%%%%%%%%%%%%%%%%%%%%%%%%%%
% Author: Xiong Yiliang
% Email: wlxiong@gmail.com
% Update: October 11, 2010
% Institution: Hong Kong Polytechnic University
% Title: Definitions and Theorems in Linear Programming
%%%%%%%%%%%%%%%%%%%%%%%%%%%%%%%%%%%%
% !TEX TS-program = pdflatex
% !TEX encoding = UTF-8 Unicode

% This is a simple template for a LaTeX document using the "article" class.
% See "book", "report", "letter" for other types of document.

\documentclass[11pt]{article} % use larger type; default would be 10pt

\usepackage[utf8]{inputenc} % set input encoding (not needed with XeLaTeX)

%%% Examples of Article customizations
% These packages are optional, depending whether you want the features they provide.
% See the LaTeX Companion or other references for full information.

%%% PAGE DIMENSIONS
\usepackage{geometry} % to change the page dimensions
\geometry{a4paper} % or letterpaper (US) or a5paper or....
% \geometry{margins=2in} % for example, change the margins to 2 inches all round
% \geometry{landscape} % set up the page for landscape
%   read geometry.pdf for detailed page layout information

\usepackage{graphicx} % support the \includegraphics command and options

%%% MATH ENVIRONMENT
\usepackage{amsmath}
\usepackage{amsthm}
\theoremstyle{plain}% default
\newtheorem{thm}{Theorem}[section]
\newtheorem{lem}[thm]{Lemma}
\newtheorem{prop}[thm]{Proposition}
\newtheorem*{cor}{Corollary}
\theoremstyle{definition}
\newtheorem{dfn}{Definition}[section]
\newtheorem{exmp}{Example}[section]
\newtheorem{prob}[exmp]{Problem}
\theoremstyle{remark}
\newtheorem*{rem}{Remark}
\newtheorem*{note}{Note}

\usepackage[parfill]{parskip} % Activate to begin paragraphs with an empty line rather than an indent

%%% PACKAGES
\usepackage{booktabs} % for much better looking tables
\usepackage{array} % for better arrays (eg matrices) in maths
\usepackage{paralist} % very flexible & customisable lists (eg. enumerate/itemize, etc.)
\usepackage{verbatim} % adds environment for commenting out blocks of text & for better verbatim
\usepackage{subfig} % make it possible to include more than one captioned figure/table in a single float
% These packages are all incorporated in the memoir class to one degree or another...

%%% HEADERS & FOOTERS
\usepackage{fancyhdr} % This should be set AFTER setting up the page geometry
\pagestyle{fancy} % options: empty , plain , fancy
\renewcommand{\headrulewidth}{0pt} % customise the layout...
\lhead{}\chead{}\rhead{}
\lfoot{}\cfoot{\thepage}\rfoot{}

%%% SECTION TITLE APPEARANCE
\usepackage{sectsty}
\allsectionsfont{\sffamily\mdseries\upshape} % (See the fntguide.pdf for font help)
% (This matches ConTeXt defaults)

%%% ToC (table of contents) APPEARANCE
\usepackage[nottoc,notlof,notlot]{tocbibind} % Put the bibliography in the ToC
\usepackage[titles,subfigure]{tocloft} % Alter the style of the Table of Contents
\renewcommand{\cftsecfont}{\rmfamily\mdseries\upshape}
\renewcommand{\cftsecpagefont}{\rmfamily\mdseries\upshape} % No bold!

%%% END Article customizations

%%% The "real" document content comes below...

\title{Definitions and Theorems in Linear Programming}
\author{Xiong Yiliang}
\date{} % Activate to display a given date or no date (if empty),
         % otherwise the current date is printed 

\begin{document}
\maketitle

\section{The Basics}

\subsection{Two special forms of linear programs}

Let us look at two special forms of linear programs. 

\begin{itemize}
\item[(I) Standard form] 
\begin{align*}
\min c^T x \\
\text{s.t. } Ax = b\\
x \geq 0
\end{align*}
in which every variable is sign-constrained and only equality constraints are allowed. 
\item[(II) Canonical form]
\begin{align*}
\min c^T x\\
\text{s.t. } Ax \geq b\\
x \geq 0
\end{align*}
a form which allows only inequality constraints and in which all variables are
sign-constrained. 
\end{itemize}

\subsection{Linear algebra and geometry}
\textbf{Conventions}. The $j$th column of an array $A$ (as a column vector) will 
be denoted $A_j$; ite $i$th row, as a \emph{row vector}, will be denoted $A^i$

A \emph{vector space} of \emph{linear space} $S$ (in $R^n$) is a nonempty 
subset of $R^n$ closed under vector addition and scalar multiplication. Vectors 
$v_1, v_2, \cdots, v_r$ are \emph{linear independent} if and only if whenever
$\sum c_i v_i = 0$ and $c_i \in R$, then $c_i = 0$ for all $i$. Vectors that are
not linearly independent are \emph{linearly dependent}. The dimension $dim(S)$ 
of a linear space $S$ is the maximum number of linearly independent vectors in $S$. 

An \emph{affine space} (in $R^n$) is the translate of a linear space. Formally, 
$A \subset R^n$ is an affine space if and only if $A = \{ t + y \mid y \in S \}$ 
for a fixed $n$-vector $t$ and linear space $S$. The dimension of $A$ is defined 
to be $dim(s)$. By extension, if $B$ is an arbitrary subset of $R^n$, $dim(S)$ 
is defined to be the smallest dimension of any affine space containing $B$. 

If $A$ is an $m\times n$ matrix, we define 
$$\text{column space} (A) = \{ Ax \mid x \in R^n\}, $$
and 
$$ rank(A) = dim(\text{column space} (A)). $$

We say that $A$ has \emph{full rank} if $rank(A)$ is its smaller dimension. 
$nullspace(S)$ is the vector space $\{ x \mid Ax = 0 \}$ and the \emph{nullity} 
of $A$ is the dimension of $nullspace(A)$. 

\begin{thm}
$rank(A)$ is simultaneously the maximum number of linearly independent rows in $A$
and the maximum number of linearly independent columns in $A$. 
\end{thm}

\begin{thm}
$rank(A) + nullity(A) = n,$ if $A$ is $m\times n$.
\end{thm}

Where $W$ is a subset of $R^n$, let the \emph{orthogonal complement} 
$W^{\perp}$ of $W$ be $\{ y \in R^n \mid y^T x = 0 \forall x \in W \}$, 
the vector space of vectors perpendicular to $W$. 

\begin{thm}
Every $x \in R^n$ can be written uniquely as $x = x^W + x^{\perp}$, where 
$x^W \in W$ and $x^{\perp} \in W^{\perp}$. 
\end{thm}

The vector $x^W$ is known as the \emph{projection} of $x$ onto $W$, while 
$x^{\perp}$ is the $projection$ of $x$ onto $W^{\perp}$. 

A set of $n$-vectors $u_1, u_2, \cdots, u_m$ is orthonormal if each has unit 
length and distinct vectors have $0$ dot product. Alternatively, 
$u_1, u_2, \cdots, u_m$ are orthonormal if and only if $u^T_k u_l = 0$ 
if $k\neq l$ and $u^T_k u_k = 1$ for all $k$. A real square matrix $U$ is 
\emph{orthogonal} if $UU^T = I$, i.e., its rows (and hence columns) are 
orthonormal. 

\begin{thm}
If $A$ is real symmetric matrix, all of $A$'s eigenvalues are real. Furthermore, 
if $A$'s eigenvalues are $\lambda_1, \lambda_2, \cdots, \lambda_n$, then 
$$ A = U \Lambda U^{-1}, $$
where $ \lambda = diag\{ \lambda_1, \lambda_2, \cdots, \lambda_n\}$ and
$U$ is an orthogonal matrix whose $i$th column is the eigenvector of $A$ 
corresponding to $\lambda_i$, normalized to have length one. 
\end{thm}

\begin{thm}
Let $A$ be real symmetric. The following are equivalent. 
\begin{itemize}
\item $A$ is positive definite.
\item All eigenvalues of $A$ are positive. 
\item $A = QQ^T$ for a nonsingular $n\times n$ real matrix $Q$. 
\end{itemize}
\end{thm}

\begin{dfn}
A \emph{hyperplane} in $R^n$ is 
$$ H = \{ x \mid a_1 x_1 + \cdots + a_n x_n = b \}, $$
where not all $a_i = 0$. It is easy to verify that $dim(H) = n - 1$. 
A \emph{halfspace} is the set $\{ x \mid a_1 x_1 + \cdots + a_n x_n \leq b \}$ 
where not all $a_i = 0$. 
\end{dfn}

We say that a set $T \subset R^n$ is bounded if there is a real $r$ such that
$\parallel x\parallel \leq r$ for all $x \in T$. 

\begin{dfn}
A \emph{polyhedron} is the intersection of finitely many halfspaces. A bounded 
nonempty polyhedron is called a \emph{polytope}. 
\end{dfn}

\begin{exmp}
The set $\{ x \mid Ax \geq b, x \geq 0\}$ is the intersection of finitely many halfspaces, 
as is $\{ x \mid Ax = b, x\geq 0\}$. If they are nonempty and bounded, they are 
polytopes. 
\end{exmp}

\subsection{Basic solutions}

\begin{dfn}
Suppose $x_1, x_2, \cdots, x_r \in R^n$, if $p = \sum^r_{i=1} \lambda_i x_i, \lambda_i \geq 0$ 
and $\sum^r_{i=1} \lambda_i = 1$, then $p$ is the \emph{convex combination} 
of $x$ and $y$. Point $p$ is the \emph{strict convex combination} of points
$ x_1, x_2, \cdots, x_r$ if $p$ is the convex combination of $ x_1, x_2, \cdots, x_r$, 
yet $p \neq x_i$ for all $i$. 
\end{dfn}

\begin{dfn}
$S \subset R^n$ is \emph{convex} if whenever $x, y \in S$, $\lambda x + (1-\lambda)y \in S$ 
for all $\lambda \in [0, 1]$. That is, the line segment between $x$ and $y$ lies in $S$. 
\end{dfn}

\begin{dfn}
If $P$ is a polyhedron, $v \in P$ is an \emph{extreme point} or \emph{vertex} 
if $v$ cannot be written as the strict convex combination of points $x, y \in P$. 
\end{dfn}

\begin{dfn}
If $A$ is an $m\times n$ matrix of rank $m$, any $m$ linearly independent columns are a
\emph{basis}. They form a basis of $A$'s column space, $R^n$. 
\end{dfn}

\textbf{Notation}. Let us use LPS to denote
\begin{align*}
\min c^T x\\
\text{s.t. } Ax = b\\
x \geq 0
\end{align*}
where $A$ is an $m\times n$ matrix of rank $m$. We use $F$ to denote the feasible set 
$\{x \mid Ax = b, x \geq 0\}$. 

\begin{thm}
In LPS, suppose that $w\in R^n$. Then $w$ is a vertex of the feasible set $F$ if and only if
$w$ is a basic feasible solution of the LPS instance. 
\end{thm}

\begin{proof}

\end{proof}

\begin{cor}
There are only finitely many vertices. 
\end{cor}

\begin{proof}

\end{proof}

\begin{dfn}
If a basic feasible solution is \emph{degenerate} if it has more than $n-m$ zeros. 
\end{dfn}

\begin{thm}
If two different bases correspond to a single basic feasible solution $v$, the $v$ is 
degenerate. 
\end{thm}

\begin{proof}

\end{proof}

\begin{lem}
Suppose $P = \{ x \in R^n \mid Ax \leq b \}$ is a polyhedron. Then $v$ is an extreme 
point of $P$ if and only if there are $n$ linearly independent constraints among the 
constraints $Ax \leq b$ that are tight at $v$. 
\end{lem}

\begin{proof}

\end{proof}

\begin{thm}
In LPS, suppose that $p\in F$. Then either the LPS instance is unbounded or there is 
a vertex $v$ of $F$ satisfying $c^T v \leq c^T p$. 
\end{thm}

\begin{proof}

\end{proof}

\begin{cor}
If in LPS $F = \emptyset$ and $c^T x \geq B$ for all $x \in F$, then there is an 
optimal vertex. Provided that the cost is bounded below, linear programming is a finite
problem! 
\end{cor}

\begin{proof}

\end{proof}

\begin{thm}
If $P = \{ x \in R^n \mid Ax \leq b \}$ is a polytope, the every $x \in P$ can be written 
as a convex combination of at most $n+1$ vertices. 
\end{thm}

\begin{proof}

\end{proof}

\subsection{The fundamental theorem of linear inequalities}

\begin{thm}
Let $a_1, \cdots, a_m, b \in R^n$ be a set of vectors. Then exactly one of 
the two following alternatives occurs:
\begin{itemize}
\item[(I)] $\exists y \in R^m_+$ such that $b = \sum^m_i y_i a_i$. 
\item[(II)] $\exists d \in R^n$ such that $d^Tb < 0$ and $d^Ta_i \geq 0$ for all 
$(i = i, \cdots, m)$. 
\end{itemize}
\end{thm}

\begin{proof}

\end{proof}

\section{The Simplex Algorithm}

\subsection{Pivoting}

\begin{thm}
Given ordered basis $B = B(1) , B(2), \cdots, B(m)$ and the basic feasible solution $x_0$
corresponding to $B$ (whose $B(i)$th component is $x_{i0}$, $1 \leq i \leq m$, let $j \in B$. 
Write $A_j = \sum_{i=1}^m x_{ij} A_{B(i)}$. Suppose $\{ i \mid x_{ij} > 0 \}$ is nonempty. 
Let $\theta_0$ denote $\min_{i:x_{ij} > 0} \frac{x_{i0}}{x_{ij}}$ and suppose 
$\theta_0 = \frac{x_{l0}}{x_{lj}}$ (with $x_{lj} > 0$). Let $ B'(i) = \left\{ 
\begin{aligned}
&B(i), &i \neq l\\
&j, &i = l
\end{aligned}
\right. $. Then $B' = B'(1), B'(2), \cdots, B'(m)$ is a basis corresponding to the basic feasible 
solution $x'_0$ whose $B'(i)$th component is $x_{i0} - \theta_0 x_{ij}$, if $i \neq l$, and whose 
$j$th component is $\theta_0$. 
\end{thm}

\begin{proof}

\end{proof}

\textbf{Potential Problems:}
\begin{itemize}
\item There might not be any $i$ such that  $x_{ij} > 0$. We can then choose any 
$\theta_0 \geq 0$ and remain feasible. 
\item $\theta_0$ may be $0$, in which case $x'_0 = x_0$ (but the ordered bases differ).
A pivot with $\theta_0 = 0$ is called \emph{degenerate}. 
\end{itemize}

\begin{note}
	The basic feasible solution $x_0$ has $n$ entries, of which at least $n-m$ are $0$. 
	If there is a tie in the min operation that determines $\theta_0$, the new basic 
	feasible solution is degenerate --- we get at least two $0$'s "at cost of one". 
\end{note}

\subsection{Tableaux}

Pivoting, a piece of cake. But which nonbasic column $j$ should we bring into the basis? 
The cost of a basic feasible solution $x_0$ with ordered basis $B(1), B(2), \cdots, B(m)$ is 
$$ \sum_{i=1}^{m} c_{B(i)} x_{i0}. $$ 
How will bringing column $j$ into the basic change the cost? We set
$$ \theta_0 = \min_{i:x_{ij}>0} \frac{x_{i0}}{x_{ij}}$$
(if the min operation is over a nonempty set) and choose $l$ such that
$$ \theta_0 = \frac{x_{l0}}{x_{lj}}$$
where $x_{lj}>0$. If column $j$ enters, $B(l)$ leaves. 
$$ \text{new cost} = \sum_{i\neq l} c_{B(i)} [x_{i0} - \theta_0 x_{ij}] + c_j \theta_0$$
(because $[x_{i0} - x_{ij}]$ will be the $B(i)$th entry of the new basic feasible solution, for 
$i \neq l$, and its $j$th entry will be $\theta_0$). 
\begin{align*}
\text{old cost} &= \sum_{i\neq l} c_{B(i)} x_{i0} + c_{B(l)} x_{l0} \\
&= \sum_{i\neq l} c_{B(i)} x_{i0} + \theta_0 c_{B(l)} x_{lj}, 
\end{align*}
since $x_{l0} = \frac{x_{l0}}{x_{lj}}\cdot x_{lj} = \theta_0 x_{lj}$. 
So
\begin{align*}
[\text{new cost} - \text{old cost}] &= \theta_0 [c_j - \sum_{i\neq l} c_{B(i)} x_{ij} - c_{B(l)} x_{lj}]\\
&= \theta_0 [c_j - \sum_i c_{B(i)} x_{ij}]. 
\end{align*}
To decrease the cost, we want $c_j - \sum_i c_{B(i)} x_{ij}$ to be negative. 

Where $z$ is the $n$-vector satisfying $z_j = \sum_i c_{B(i)} x_{ij}$, let $\bar{c} = c - z$, 
the $n$-vector of \emph{modified costs}. It is then profitable to bring $j$ into the basis if 
$\bar{c}_j<0$ . In this case the cost increase by $\theta_0 \bar{c}_j<0$, which is nonpositive. 

\begin{lem}
Let $X$ be the tableau (without column $0$) after $s$ pivots, where $A$ was the original matrix. 
Let columns $B(1), B(2), \cdots, B(m)$ be the current ordered basis where column $B(i)$ of $X$ 
is $e_i$. Let $B$ denote the $m\times m$ matrix whose $i$th column is $A_{B(i)}$. 
Then $X = B^{-1} A$. Furthermore, the $0$th column is exactly $B^{-1}b$. 
\end{lem}

\begin{proof}

\end{proof}

\begin{thm}
If $\bar{c} = c - z \geq 0$ at basic feasible solution $x_0$, then $x_0$ is optimal. 
\end{thm}

\begin{proof}

\end{proof}

\begin{thm}
For all $i$, $\bar{c}_{b(i)} = 0$, i.e., the modified cost of a basic column is $0$. 
\end{thm}

\begin{proof}

\end{proof}

\begin{lem}
The $(0,0)$ entry will be always hold the negative of the cost of the current basic 
feasible solution, 
$$ - \sum_{i=1}^m x_{i0} c_{B(i)}$$
and the $(0,j)$ entry will always hold $\bar{c}_j$ (for $j \geq 1$), 
$$ \bar{c}_j = c_j - \sum_{i=1}^m x_{ij} c_{B(i)}$$
if whenever we pivot we perform a row operation so as to zero out the entry atop
the entering column. 
\end{lem}

\begin{proof}

\end{proof}

Let us review pivoting. 
\begin{enumerate}
	\item Find a column $j$ whose row-$0$ entry is negative. 
	\begin{itemize}
		\item Find $j$ such that $c_j - \sum_i c_{B(i)} x_{ij}$ is negative. 
		\item A greedy approach --- find the steepest decreasing direction: 
			\[ j^* = \arg \min_j \{ c_j - \sum_i c_{B(i)} x_{ij} \} \]
	\end{itemize}
	\item Compute $\theta_0 = \min_{i:x_{ij}>0} \frac{x_{i0}}{x_{ij}}$. 
	(Assume, for now, that some $x_{ij} > 0$.) 
	\begin{itemize}
		\item If $x_{ij} \leq 0$ for all $i$, then the LP is unbounded
		(we can always move along the decreasing direction, 
		and the objective goes to $-\infty$). 
		\item Otherwise, there is a maximum step $\theta_0$. To maintain 
		the feasibility, we cannot move any further than $\theta_0$. 
	\end{itemize}
	\item Find an $l$ such that $\theta_0 = \frac{x_{l0}}{x_{lj}}$ 
	(amd $x_{lj} > 0$). 
\end{enumerate}

Combine the first two steps and make \emph{one big jump}. 
\begin{itemize}
	\item We can determine the entering column and leaving column simultaneously 
	by solving a nested minimization problem: 
	\begin{align*}
	\Delta = 
	\min_j \left\{ \left[ \min_{i:x_{ij}>0} \left\{ \frac{x_{i0}}{x_{ij}}\right\} \right]
	\cdot \left[ c_j - \sum_i c_{B(i)} x_{ij}\right] \right\}. 
	\end{align*}
	\item Let $i^*$ and $j^*$ denote the solution of above problem. 
	Now make a sequence of row operations to bring column $j^*$ into the basis 
	and make $i^*$ leave the basis, then we decrease the objective by $\Delta$. 
	\begin{itemize}
		\item $\Delta$ is the maximum decrease we can achieve by moving to 
		an adjacent vertex of $v_0$, where $v_0$ is the vertex corresponding to 
		current basic solution $x_0$. 
	\end{itemize}
\end{itemize}

\begin{thm}
If $\bar{c}_j < 0$, and $x_{ij} \leq 0$ for all $i$, then the LPS instance is unbounded. 
\end{thm}

\begin{proof}

\end{proof}

\textbf{Summary.} For a given linear program: 
\begin{align*}
\min c^T x\\
\text{s.t. } Ax = b\\
x \geq 0
\end{align*}

The initial tableau of the LP problem is: 
\begin{center}
\begin{tabular}{|l|l|}
\hline
0 & $c$ \\
\hline
$b$ & $A$ \\
\hline
\end{tabular}
\end{center}

Suppose $B$ is a basis and the tableau is then: 
\begin{center}
\begin{tabular}{|l|l|}
\hline
-$c_B^T B^{-1} b$ & $c - c_B^T B^{-1} A$ \\
\hline
$B^{-1} b$ & $B^{-1} A$ \\
\hline
\end{tabular}
\end{center}

\section{Duality}

\subsection{The Dual}
The primal and dual problems: 
\begin{center}
\begin{tabular}{l|l l}
 & Primal & Dual \\
 \hline
 & min $c^T x$ & max $y^T b$\\
 row $i$ & $\sum_j a_{ij} x_j = b_i$ & $y_i \in R$\\
 row $i$ & $\sum_j a_{ij} x_j \geq b_i$ & $y_i \geq R$\\
 var $j$ & $x_j \in R$ & $\sum_i y_i a_{ij} = c_j$\\
 var $j$ & $x_j \geq 0$ & $\sum_i y_i a_{ij} \leq c_j$\\
\end{tabular}
\end{center}

\begin{thm}
The dual of the dual is the primal. 
\end{thm}

\subsection{The Duality Theorem}

\begin{thm}
If $w, u$ is a primal/dual feasible pair, $ c^T w \geq u^T b$. 
\end{thm}

\begin{proof}

\end{proof}

\begin{thm}
If a linear program has an optimal solution, so does its dual, and their optimal 
costs are identical. 
\end{thm}

\begin{proof}

\end{proof}

\begin{cor}
If we run Simplex on LPS, at termination $(B^{-1})^T c_B is$ a dual optimal solution
(if an optimal point exists). 
\end{cor}

\begin{cor}
Exactly one of these three cases occurs: 
\begin{itemize}
\item Primal and dual are both infeasible. 
\item One is unbounded and the other is infeasible. 
\item Both have optimal points. 
\end{itemize}
\end{cor}

\begin{cor}
Suppose that the leftmost $m$ columns of $A$ are an identity matrix. If we run Simplex
on LPS and it terminates at an optimal point, at termination the $m$-vector $u^*$ given 
by $u^*_i = c_i - \bar{c}_i$ is dual optimal. 
\end{cor}

\subsection{Complementary Slackness}

\begin{thm}
Let $P$ be a linear program in general form:
\begin{align*}
\min c^T x \\
\text{s.t. } A^i x &= b_i, 1 \leq i \leq h \\
A^i  x &\geq b_i, h+1 \leq i \leq m \\
x_j &\geq 0, 1 \leq j \leq l \\
x_j &\in R, l+1 \leq j \leq n. 
\end{align*}
Let its dual be $D$: 
\begin{align*}
\max b^T y \\
\text{s.t. } A_j^T y &\leq c_j, 1 \leq j \leq l \\
A_j^T y &= c_j, l+1 \leq j \leq n \\
y_i &\in R, 1 \leq i \leq h \\
y_i &\geq 0, h+1 \leq i \leq m
\end{align*}
Let $w$ be primal feasible and let $u$ be dual feasible. Then $w$ is primal optimal 
and $u$ is dual optimal if and only if 
$$ (A^i w - b_i) u = 0 \text{ for } i = 1, 2, \cdots, m$$
and 
$$ (A_j^T - c_j) w = 0 \text{ for } j = 1, 2, \cdots, n. $$
\end{thm}

\subsection{Farkas's Lemma}

\begin{thm}
$Ax \leq b$ has a solution if and only if there is no nonnegative vector $y$ satisfying 
$A^T y = 0$ and $ b^T y < 0$. 
\end{thm}

\begin{proof}

\end{proof}

\subsection{The Economic Interpretation of the Dual}

\begin{thm}
Consider a primal problem $P$: 
\begin{align*}
\min c^T x \\
Ax = b \\
x \geq 0
\end{align*}
Suppose that $P$ has at least one nondegenerate basic optimal solution. Then there is
a unique $u^* \in R^m$ which is optimal in the dual of $P$. Furthermore, there is 
a positive $\delta$ with the following property. 

Suppose $t \in R^m$ and $\mid t_i \mid \leq \delta$ for all $i$. Then 
\begin{align*}
\min c^T x \\
Ax &= b + t \\
x &\geq 0
\end{align*}
has a optimal solution and its value is $C^* + (u^*)^T t$, where $C^*$ is the optimal 
cost of $P$ and $u^*$ is dual optimal. 
\end{thm}

\begin{proof}

\end{proof}


\end{document}
