%%%%%%%%%%%%%%%%%%%%%%%%%%%%%%%%%%%%%%%%%%%%%%%
% Author: Xiong Yiliang
% Email: wlxiong@gmail.com
% Update: December 16, 2010
% Institution: Hong Kong Polytechnic University
% Title: Note on consumer’s maximization problem
%%%%%%%%%%%%%%%%%%%%%%%%%%%%%%%%%%%%%%%%%%%%%%%
% !TEX TS-program = pdflatex
% !TEX encoding = UTF-8 Unicode

\documentclass[12pt,a4paper]{article}

\usepackage{amsmath}	% just math
\usepackage{amssymb}	% allow blackboard bold (aka N,R,Q sets)
\usepackage{amsthm}	% allow blackboard bold (aka N,R,Q sets)
\linespread{1.2}	% double spaces lines

%% page layout 
\textwidth 6.5truein  % These 4 commands define more efficient margins
\textheight 9.5truein
\oddsidemargin 0.0in
\topmargin -0.6in

\parindent 0pt	% let's not indent paragraphs
\parskip 5pt  % Also, a bit of space between paragraphs

%% bibliography
\usepackage{bibentry}

%%% MATH ENVIRONMENT
\usepackage{amsmath}
\usepackage{amsthm}
\theoremstyle{plain}% default
\newtheorem{thm}{Theorem}[section]
\newtheorem{lem}[thm]{Lemma}
\newtheorem{prop}[thm]{Proposition}
\newtheorem*{cor}{Corollary}
\theoremstyle{definition}
\newtheorem{dfn}{Definition}[section]
\newtheorem{exmp}{Example}[section]
\newtheorem{prob}[exmp]{Problem}
\theoremstyle{remark}
\newtheorem*{rem}{Remark}
\newtheorem*{note}{Note}

% I write partial derivatives a lot, it's useful to make a new command:
\newcommand{\partfrac}[2]{\frac{\partial #1}{\partial #2}}
% Then, writing a partial is as simple as $\partfrac{J}{x}$.
% A one-line partial derivative command: 
\newcommand{\partline}[2]{\partial #1 \left/ \partial #2 \right.}

% Smilarily, the commands for derivative is: 
\newcommand{\devfrac}[2]{\frac{d #1}{d #2}}
\newcommand{\devline}[2]{d #1 \left/ d #2 \right.}

\begin{document}
\begin{flushright}
\linespread{1}	% single spaces lines
\small \normalsize %% dumb, but have to do this for the prev to work
Yiliang Xiong \\
\today
\end{flushright}

%% The context begins here. 

\begin{rem}
Note on consumer's maximization problem: excerpted from \bibentry{small:1981}
\end{rem}

Suppose that a consumer has a \emph{twice differentiable}, \emph{strictly quasi-concave} 
utility function $u$ defined over the commodities $x_n$ and $x_1$, where $x_n$ is taken to 
be the \emph{numeraire}. The two good case is chosen for convenience; the results generalize 
easily to an arbitrary number of commodities. The consumer maximizes (\emph{direct utility
function}): 
\begin{equation}\label{util}
	u = u(x_n, x_1)	
\end{equation}
subject to the budget and nonnegative constraints 
\begin{align} 
	\label{budget}
	x_n + p_1 x_1 &= y, \\
	x_j \geq 0, j &= n,1,	
\end{align}
where $p_1$ is the price of good $1$, and $y$ is income. 

Maximization of (\ref{util}) subject to (\ref{budget}) yields the \emph{ordinary demand functions} 
(\emph{Marshallian demand})
\begin{equation} \label{demand}
	x_j = x_j(p_1, y), j = n,1. 
\end{equation}

We assume that an interior solution ($x_j>0$) with respect to both goods. Substituting 
(\ref{demand}) into (\ref{util}) we define the consumer's \emph{indirect utility function} 
\begin{equation} \label{indirect_util}
	v(p_1, y) = u(x_n(p_1,y), x_1(p_1,y))
\end{equation}
which is known to satisfy \emph{Roy's Identity}: 
\begin{equation} \label{roy}
	-\frac{\partial v \left/ \partial p_1 \right. }%
		  {\partial v \left/ \partial y \right. } = x_1
\end{equation}

\begin{note}
	Roy's Identity follows from comparing the total differentials of (\ref{budget}) 
	and (\ref{indirect_util}) and using the first-order maximum conditions of the consumer's 
	maximum problem (\ref{util}) and (\ref{budget}).
\end{note}

\begin{proof}\label{pf:roy}
	First we apply the chain rule to (\ref{indirect_util}) to derive its total differential: 
	\begin{align} 
		\label{total_dev_p}
		\partfrac{v}{p_1} &= \partfrac{u}{x_n} \cdot \partfrac{x_n}{p_1} + 
							 \partfrac{u}{x_1} \cdot \partfrac{x_1}{p_1} \\
		\label{total_dev_y}
		\partfrac{v}{y} &= \partfrac{u}{x_n} \cdot \partfrac{x_n}{y} + 
						   \partfrac{u}{x_1} \cdot \partfrac{x_1}{y} 
	\end{align}
	
	The Lagrangian function of the consumer's maximum problem is: 
	\begin{equation} \label{Lagrangian}
		L = u + \lambda (y - x_n - p_1 x_1)
	\end{equation}
	The first-order condition is: 
	\begin{align} \label{first_order}
		\partfrac{L}{x_1} &= \partfrac{u}{x_1} - \lambda p_1 = 0 \\
		\partfrac{L}{x_n} &= \partfrac{u}{x_n} - \lambda = 0 
	\end{align}
	
	Substitute (\ref{first_order}) into (\ref{total_dev_p}) and (\ref{total_dev_y}): 
	\begin{align} 
		\label{total_dev_p_opt}
		\partfrac{v}{p_1} &= \lambda \cdot \partfrac{x_n}{p_1} + 
							 \lambda p_1 \cdot \partfrac{x_1}{p_1} \\
		\label{total_dev_y_opt}
		\partfrac{v}{y} &= \lambda \cdot \partfrac{x_n}{y} + 
						   \lambda p_1 \cdot \partfrac{x_1}{y}
	\end{align}
	Then we have, 
	\begin{equation} 
		\label{roy_frac}
		\frac{\partline{v}{p_1}}{\partline{v}{y}} = 
		\frac{\partline{x_n}{p_1} + p_1 \cdot \partline{x_1}{p_1}}%
			 {\partline{x_n}{y} + p_1 \cdot \partline{x_1}{y}}. 
	\end{equation}
	
	To further simplify (\ref{roy_frac}), use the total differential of (\ref{budget}): 
	\begin{align}
		\label{total_dev_budget_p}
		\partfrac{x_n}{p_1} + x_1 + p_1 \cdot \partfrac{x_1}{p_1} &= 0 \\
		\label{total_dev_budget_y}
		\partfrac{x_n}{y} + p_1 \cdot \partfrac{x_1}{y} &= 1
	\end{align}
	Substitute (\ref{total_dev_budget_p}) and (\ref{total_dev_budget_y}) into (\ref{roy_frac}), 
	\begin{equation}
		\frac{\partline{v}{p_1}}{\partline{v}{y}} = -x_1
	\end{equation}
	Roy's Identity is obtained. 
\end{proof}

Provided that the direct utility function is \emph{strictly increasing} in $x_n$ and is 
\emph{nondecreasing} in $x_1$, $v$ is \emph{monotonic increasing} in $y$ and can 
therefore be inverted to yield the \emph{expenditure function} 
\begin{equation} \label{expend}
	y = e(p_1, u). 
\end{equation}

The function $e$ therefore indicates how much income is required to achieve the utility level 
$u$ when the price of good $1$ is $p_1$; it satisfies 
\begin{equation} \label{util_identity}
	u = v(p_1, e(p_1,u)). 
\end{equation}

Now suppose that the price of the first good changes from $p_1^0$ to $p_1^f$. By definition, 
the compensating variation associated with the price change is 
\begin{equation} \label{delta_e}
	\Delta e = e(p_1^f, u^0) - e(p_1^0, u^0). 
\end{equation}

This expression shows the amount of income the consumer must be given to make him as well off 
at price $p_1^f$ as at $p_1^0$. 

The problem now is to express (\ref{delta_e}) in terms of the \emph{compensated demand function} 
(\emph{Hicksian demand}) for $x_1$, which is defined by 
\begin{equation} \label{compensated_demand}
	x_1^c (p_1, u) = x_1 (p_1, e(p_1, u)). 
\end{equation}

This is done by applying \emph{Shephard's Lemma}: 
\begin{equation} \label{shephard}
	\frac{\partial e(p_1, u)}%
		 {\partial p_1} = x_1^c (p_1, u), 
\end{equation}
where $u$ is an arbitrary selected utility level. 

\begin{note}
	Shephard's Lemma follows from differentiating (\ref{util_identity}) with respect to $p_1$, 
	and applying Roy's Identity. 
\end{note}

\begin{proof}\label{pf:shephard}
	For given utility $u$, differentiate (\ref{util_identity}) with respect to $p_1$: 
	\begin{equation}
		\partfrac{u}{p_1} = \partfrac{v}{p_1} + \partfrac{v}{y} \cdot \partfrac{e}{p_1}
	\end{equation}
	Since $u$ is fixed, thus we have $\partfrac{u}{p_1} = 0$. Rearrange the above equation: 
	\begin{equation}
		\label{expned_part_p}
		\partfrac{e}{p_1} = -\frac{\partline{v}{p_1}}{\partline{v}{y}} 
	\end{equation}
	Applying Roy's Identity (\ref{roy}) to (\ref{expned_part_p}): 
	\begin{equation}
		\partfrac{e}{p_1} = x_1
	\end{equation}
	Shephard's Lemma is proved. 
\end{proof}

The equation (\ref{shephard}) gives the compensating variation for an infinitesimal change 
in $p_1$; to find the compensating variation for a finite change, 
(\ref{shephard}) is integrated: 
\begin{equation} \label{e_difference}
	e(p_1^f, u^0) - e(p_1^0, u^0) = 
		\int_{p_1^0}^{p_1^f} x_1^c (p_1, u^0) d p_1. 
\end{equation}

This gives the basic result mentioned at the start of this section; the compensating variation 
of a price change is the area to the left of the compensated demand curve. 

\begin{rem}
	Measuring price-induced \emph{utility changes} as areas to the left of the appropriate 
	compensated demand curves. 
\end{rem}

% add QED symbol abbreviation
% -->ab qed. \hfill$\square$ 

\bibliographystyle{plain}
\nobibliography{/Users/xiongyiliang/Documents/BibDesk/classics.bib}

\end{document}
